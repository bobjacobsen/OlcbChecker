\documentclass[11pt]{article}
\usepackage{geometry}                % See geometry.pdf to learn the layout options. There are lots.
\geometry{letterpaper}                   % ... or a4paper or a5paper or ... 
%\geometry{landscape}                % Activate for for rotated page geometry
\usepackage[parfill]{parskip}    % Activate to begin paragraphs with an empty line rather than an indent
\usepackage{graphicx}
\usepackage{amssymb}
\usepackage{epstopdf}
\usepackage{hyperref}
\DeclareGraphicsRule{.tif}{png}{.png}{`convert #1 `dirname #1`/`basename #1 .tif`.png}

% fornat header abd footer
\usepackage{fancyhdr}
\fancyhf{}
\fancyfoot[L]{\footnotesize{Copyright 2022-2024. All rights reserved. 
        This OpenLCB document is licensed under the 
        \newline
        Creative Commons Attribution-ShareAlike 4.0 International License (CC BY-SA 4.0). 
        \newline
        See https://openlcb.org/licensing for more information.
    }
    \hfill \thepage
}
\renewcommand\headrulewidth{0pt}
\setlength{\headheight}{13.59999pt}

\fancypagestyle{firststyle}
{
   \fancyhf{}
    \fancyfoot[L]{\footnotesize{Copyright 2022-2024. All rights reserved. 
        This OpenLCB document is licensed under the 
        \newline
        Creative Commons Attribution-ShareAlike 4.0 International License (CC BY-SA 4.0). 
        \newline
        See https://openlcb.org/licensing for more information.
        }
        \hfill \thepage
    }
   \renewcommand{\headrulewidth}{0pt} % removes horizontal header line
}

\pagestyle{fancyplain}
% The name is put in the header by the titleandheader command below

% common header content elements
%\author{The OpenLCB Group}
\date{\today}                                         % Activate to display a given date or no date


% specify the text for the title and footer
\newcommand*{\titleandheader}[1]{
    \title{#1}
    \fancyhead[L]{\footnotesize{#1}}
}

% to format XML tags in proper angle brackets <>
\newcommand*{\xml}[1]{\texttt{<#1>}}

% macro for inserting common introduction text
% takes hyperlink to a standard and the standard name as arguments
\newcommand{\introductionCaveats}[2]{

    \section{Introduction}

    This note documents the procedure for checking an OpenLCB implementation against the 
    \href{#1}
    {#2}.

    The checks are traceable to specific sections of the Standard.

    The checking assumes that the Device Being Checked (DBC) is being exercised by other
    nodes on the message network, 
    e.g. is responding to enquiries from other parts of the message network.
}

% macro for inserting the procedure section
\newcommand{\checkProcedure}[1]{
%    Select ``#1" in the program, 
%    then select each section below in turn.  Follow the prompts
%    for when to reset/restart the node and when to check 
%    outputs against the node documentation.
}

% macro deciding whether the pipset footnote should appear or not
\newcommand{\pipsetFootnote}{
%\footnote{Using the -p option or setting the checkpip default value False will skip this check.}
}


\titleandheader{Checking the OpenLCB Datagram Transport Protocol}

\begin{document}
\maketitle
\thispagestyle{firststyle}

\introductionCaveats
{https://nbviewer.org/github/openlcb/documents/blob/master/standards/DatagramTransportS.pdf}
{Datagram Transport Standard}.

\section{Datagram Transport Procedure}

\checkProcedure{Datagram Checking}

Note that this process is unable to check datagrams from the DBC to the checker.
There is no standard-defined way to elicit those.  They will be checked as part of the 
Memory Configuration Protocol checking, if applicable.

A node which does not self-identify in PIP that it supports
datagram transfer will be considered to have passed these checks.
\pipsetFootnote

\subsection{Datagram Reception}

This checks the messages in Standard sections 4.1 and 4.2 or 4.3, 
including the error code constraints in Standard section 4.3.1
\footnote{This section references the Message Transport Standard, 
section 3.5.5}
and the interactions in sections 6.1 or 6.2.

The checker will send a datagram to the DBC. The DBC in turn will 
either accept the datagram (sections 4.2 and 6.1) or 
reject it (sections 4.3 and 6.2). Either response is acceptable.

This section's checks cover:

\begin{enumerate}
\item If the datagram is accepted,
    \begin{enumerate}
    \item That the Datagram OK message is received,
    \item That the Datagram OK message has proper source and destination node IDs, 
    \item that the Datagram OK message contains exactly 1 byte of flags,
    \item that the 0x70 bits of the flags are zeros.
    \end{enumerate}
\item if the datagram is rejected, 
    \begin{enumerate}
    \item That the Datagram Rejected message is received,
    \item That the Datagram Rejected message has proper source and destination node IDs, 
    \item that the Datagram Rejected message contains at least two bytes of error code,
    \item that the 0xF000 bits of the error code are either 0x1000 or 0x2000,
    \item that the 0x0F00 bits of the error code are zeros.
    \end{enumerate}
\end{enumerate}

This process is repeated for datagrams of length 1, 10 and 72 bytes to exercise the full
range of datagram framing.


\end{document}  
