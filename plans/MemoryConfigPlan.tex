\documentclass[11pt]{article}
\usepackage{geometry}                % See geometry.pdf to learn the layout options. There are lots.
\geometry{letterpaper}                   % ... or a4paper or a5paper or ... 
%\geometry{landscape}                % Activate for for rotated page geometry
\usepackage[parfill]{parskip}    % Activate to begin paragraphs with an empty line rather than an indent
\usepackage{graphicx}
\usepackage{amssymb}
\usepackage{epstopdf}
\usepackage{hyperref}
\DeclareGraphicsRule{.tif}{png}{.png}{`convert #1 `dirname #1`/`basename #1 .tif`.png}

% fornat header abd footer
\usepackage{fancyhdr}
\fancyhf{}
\fancyfoot[L]{\footnotesize{Copyright 2022-2024. All rights reserved. 
        This OpenLCB document is licensed under the 
        \newline
        Creative Commons Attribution-ShareAlike 4.0 International License (CC BY-SA 4.0). 
        \newline
        See https://openlcb.org/licensing for more information.
    }
    \hfill \thepage
}
\renewcommand\headrulewidth{0pt}
\setlength{\headheight}{13.59999pt}

\fancypagestyle{firststyle}
{
   \fancyhf{}
    \fancyfoot[L]{\footnotesize{Copyright 2022-2024. All rights reserved. 
        This OpenLCB document is licensed under the 
        \newline
        Creative Commons Attribution-ShareAlike 4.0 International License (CC BY-SA 4.0). 
        \newline
        See https://openlcb.org/licensing for more information.
        }
        \hfill \thepage
    }
   \renewcommand{\headrulewidth}{0pt} % removes horizontal header line
}

\pagestyle{fancyplain}
% The name is put in the header by the titleandheader command below

% common header content elements
%\author{The OpenLCB Group}
\date{\today}                                         % Activate to display a given date or no date


% specify the text for the title and footer
\newcommand*{\titleandheader}[1]{
    \title{#1}
    \fancyhead[L]{\footnotesize{#1}}
}

% to format XML tags in proper angle brackets <>
\newcommand*{\xml}[1]{\texttt{<#1>}}

% macro for inserting common introduction text
% takes hyperlink to a standard and the standard name as arguments
\newcommand{\introductionCaveats}[2]{

    \section{Introduction}

    This note documents the procedure for checking an OpenLCB implementation against the 
    \href{#1}
    {#2}.

    The checks are traceable to specific sections of the Standard.

    The checking assumes that the Device Being Checked (DBC) is being exercised by other
    nodes on the message network, 
    e.g. is responding to enquiries from other parts of the message network.
}

% macro for inserting the procedure section
\newcommand{\checkProcedure}[1]{
%    Select ``#1" in the program, 
%    then select each section below in turn.  Follow the prompts
%    for when to reset/restart the node and when to check 
%    outputs against the node documentation.
}

% macro deciding whether the pipset footnote should appear or not
\newcommand{\pipsetFootnote}{
%\footnote{Using the -p option or setting the checkpip default value False will skip this check.}
}


\title{Checking the OpenLCB Memory Configuration Standard}

\begin{document}
\maketitle

\introductionCaveats
    {https://nbviewer.org/github/openlcb/documents/blob/master/standards/MemoryConfigurationS.pdf}
    {Memory Configuration Standard}

\section{Memory Configuration Procedure}

\checkProcedure{Memory Checking}

A node which does not self-identify in PIP that it supports
Memory Configuration will be considered to have passed these checks.
\pipsetFootnote

This plan does not check:
\begin{enumerate}
\item Stream-based memory configuration operations
\item Write and Write Under Mask to configuration memory.  
There's no generally-compatible way to do that.
You can't even assume that a read / write / read sequence will return you to the original
state, because nodes may react when information is written to them.
\footnote{See the ACDI checks in the CDI check plan for another possibility.}
\item The Reinitialize/Factory Reset Command and the Get Unique ID Command
because these permanently change the state of the node being checked.
\end{enumerate}

\subsection{Configuration Options checking}

This section checks the messages and interaction in Standard sections 4.13 and 4.14.

The checker sends a Get Configuration Options Command datagram.  It then checks:
\begin{enumerate}
\item That the datagram reply is received.
\item That a Get Configurations Options Reply datagram is received,
\item The reply datagram i sat least six bytes long,
\item That the reserved bits in bytes 2, 3 and 4 are zero.
\end{enumerate}

\subsection{Address Space Information checking}

This section checks the messages and interaction in Standard sections 4.15 and 4.16.

Using the common\footnote{See section 4.2}
0xFD and 0xFF memory configuration spaces,
\footnote{Note that 0xFE, the all-memory space, is considered optional here. The 0xFF CDI 
space check is omitted if CDI is not found in PIP.}, 
the checker sends a Get Configuration Options Command datagram.  It then checks:
\begin{enumerate}
\item That the Datagram Received OK is received,
\item That a Get Address Space Information Reply datagram is received,
\item The reply datagram is at least eight bytes long,
\item That the reply datagram refers to the same memory space as the request,
\item That the reserved bits in byte 7 are zero.
\item That if the low bit of byte 7 is 1, there's a low address present in bytes 8-11.
\end{enumerate}

If the reply comes back indicating that that space is not present, 
a warning is issued, but that part of the check is assumed to pass.

\subsection{Read Operations checking}

This section checks the messages and interaction in Standard sections 4.4 and 4.5.

The checker sends Read Commands to space 0xFD, the memory configuration space.

\begin{enumerate}
\item A read of 64 bytes from address 0,
\item A read of 10 bytes from address 0,
\item A read of 2 bytes from address 0,
\end{enumerate}

Each of these are done once with the address space in byte 1 
and once with the address space in byte 6.

For each, the checker checks:
\begin{enumerate}
\item The Datagram Received OK has the Reply Pending bit set,
\item The reply datagram is long enough to be checked,
\item It's a Read Reply datagram,
\item The error bit is not set,
\item The space matches (in either form),
\item The starting address matches,
\item The right number of data bytes are returned.
\end{enumerate}

\subsection{Lock/Reserve checking}

This section checks the messages and interaction in Standard section 4.17 and 4.18.

The checker requests that the node be reset/restarted then 

\begin{enumerate}
\item Sends a Lock/Reserve Command with node id zero
    and checks that a Lock/Reserve Reply is received with contents zero.
\item Sends a Lock/Reserve Command with node id A
    and checks that a Lock/Reserve Reply is received with contents A.
\item Sends a Lock/Reserve Command with node id B
    and checks that a Lock/Reserve Reply is received with contents A.
\item Sends a Lock/Reserve Command with zero 
    and checks that a Lock/Reserve Reply is received with contents zero.
\item Sends a Lock/Reserve Command with node id B
    and checks that a Lock/Reserve Reply is received with contents B.
\item Sends a Lock/Reserve Command with node id A
    and checks that a Lock/Reserve Reply is received with contents B.
\item Sends a Lock/Reserve Command with node id A
    and checks that a Lock/Reserve Reply is received with contents B.
\item Sends a Lock/Reserve Command with zero 
    and checks that a Lock/Reserve Reply is received with contents zero.
\item Sends a Lock/Reserve Command with zero 
    and checks that a Lock/Reserve Reply is received with contents zero.

\end{enumerate}

\subsection{Reset/Reboot checking}

This section checks the message and interaction in Standard section 4.24 Reset/Reboot Command.

The checker sends a Reset/Reboot Command and then checks that a Node Initialization 
Complete message is received to indicate a reboot.  This may or may not have been
preceded by a Datagram Received OK response.

\end{document}  
