\documentclass[11pt]{article}
\usepackage{geometry}                % See geometry.pdf to learn the layout options. There are lots.
\geometry{letterpaper}                   % ... or a4paper or a5paper or ... 
%\geometry{landscape}                % Activate for for rotated page geometry
\usepackage[parfill]{parskip}    % Activate to begin paragraphs with an empty line rather than an indent
\usepackage{graphicx}
\usepackage{amssymb}
\usepackage{epstopdf}
\usepackage{hyperref}
\DeclareGraphicsRule{.tif}{png}{.png}{`convert #1 `dirname #1`/`basename #1 .tif`.png}

% fornat header abd footer
\usepackage{fancyhdr}
\fancyhf{}
\fancyfoot[L]{\footnotesize{Copyright 2022-2024. All rights reserved. 
        This OpenLCB document is licensed under the 
        \newline
        Creative Commons Attribution-ShareAlike 4.0 International License (CC BY-SA 4.0). 
        \newline
        See https://openlcb.org/licensing for more information.
    }
    \hfill \thepage
}
\renewcommand\headrulewidth{0pt}
\setlength{\headheight}{13.59999pt}

\fancypagestyle{firststyle}
{
   \fancyhf{}
    \fancyfoot[L]{\footnotesize{Copyright 2022-2024. All rights reserved. 
        This OpenLCB document is licensed under the 
        \newline
        Creative Commons Attribution-ShareAlike 4.0 International License (CC BY-SA 4.0). 
        \newline
        See https://openlcb.org/licensing for more information.
        }
        \hfill \thepage
    }
   \renewcommand{\headrulewidth}{0pt} % removes horizontal header line
}

\pagestyle{fancyplain}
% The name is put in the header by the titleandheader command below

% common header content elements
%\author{The OpenLCB Group}
\date{\today}                                         % Activate to display a given date or no date


% specify the text for the title and footer
\newcommand*{\titleandheader}[1]{
    \title{#1}
    \fancyhead[L]{\footnotesize{#1}}
}

% to format XML tags in proper angle brackets <>
\newcommand*{\xml}[1]{\texttt{<#1>}}

% macro for inserting common introduction text
% takes hyperlink to a standard and the standard name as arguments
\newcommand{\introductionCaveats}[2]{

    \section{Introduction}

    This note documents the procedure for checking an OpenLCB implementation against the 
    \href{#1}
    {#2}.

    The checks are traceable to specific sections of the Standard.

    The checking assumes that the Device Being Checked (DBC) is being exercised by other
    nodes on the message network, 
    e.g. is responding to enquiries from other parts of the message network.
}

% macro for inserting the procedure section
\newcommand{\checkProcedure}[1]{
%    Select ``#1" in the program, 
%    then select each section below in turn.  Follow the prompts
%    for when to reset/restart the node and when to check 
%    outputs against the node documentation.
}

% macro deciding whether the pipset footnote should appear or not
\newcommand{\pipsetFootnote}{
%\footnote{Using the -p option or setting the checkpip default value False will skip this check.}
}


\titleandheader{Checking the OpenLCB Event Transport Protocol Standard}

\begin{document}
\maketitle
\thispagestyle{firststyle}

\introductionCaveats
    {https://nbviewer.org/github/openlcb/documents/blob/master/standards/EventTransportS.pdf}
    {Event Transport Standard}.

\section{Event Transport Procedure}

\checkProcedure{Event Checking}

A node which does not self-identify in PIP that it supports
Event Transfer will be considered to have passed these checks.
\pipsetFootnote

A node which does self-identify in PIP that it supports 
Event Transfer is expected to consume or produce at least 
one event.  The checks are structured to check for that.

\textbf{Note:}  Proper handling of known events should be addressed.

\textbf{Note:}  This does not address the proper use of Unique IDs for Event IDs.

\subsection{Identify Events Addressed}

This section checks the addressed interaction in Standard section 6.2 and
the message formats in Standard section 4.3 through 4.8.

The check starts by sending an Identify Events message addressed to the DBC.
It then checks

\begin{enumerate}
\item That one or more Producer Identified, Producer Range Identified, 
        Consumer Identified and/or Consumer Range Identified messages are returned,
\item That those show the DBC node as their source.
\end{enumerate}

During this process,  any received PCER messages with the DBC as their source are compared against
previously received Producer Identified and Producer Range Identified messages.
If the Event ID carried by the PCER message has not been identified by 
a previous message, the check fails.

\subsection{Identify Events Global}

This section checks the unaddressed (global) interaction in Standard section 6.2 and
the message formats in Standard section 4.3 through 4.8.

The check starts by sending an Identify Events unaddressed (global) message.

While acquiring the replies to the global message,
any received PCER messages with the DBC as their source are compared against
previously received Producer Identified and Producer Range Identified messages.
If the Event ID carried by the PCER message has not been identified by 
a previous message, the check fails.

It then checks

\begin{enumerate}
\item That one or more Producer Identified, Producer Range Identified, 
        Consumer Identified and/or Consumer Range Identified messages are returned,
\item That those show the DBC node as their source,
\item That these identify the same events and ranges produced and consumed as the 
        addressed form of the Identify Events message.

        While doing the comparison, any Producer Identified or Consumer Identified
        messages that lie within the range defined by a previous Producer Range 
        Identified or Consumer Range Identified message, respectively, are ignored
        as redundant; those do not need to exactly match.
        \footnote{Fast clocks, for example, may identify the full range
                    of events they produce, then also identify specific
                    time-varying events within those ranges.}
\end{enumerate}

\subsection{Identify Producer}

This section checks the interaction in Standard section 6.3, and
the message formats in Standard section 4.5 through 4.7.

The check proceeds by sending an Identify Producer messages for each of
the zero or more individual event IDs identified as being produced by an Identify Events message 
addressed to the DBC. If there are none of these, this check passes. If there
are one or more, it then checks:

\begin{enumerate}
\item That exactly one Producer Identified reply is received
    for each Identify Producers message sent.
\item That those show the DBC node as their source,
\item That these identify the same event ID as the corresponding Identify message.
\end{enumerate}

Other messages received during this test are ignored.

\subsection{Identify Consumer}

This section checks the interaction in Standard section 6.4, and
the message formats in Standard section 4.2 through 4.4.

The check proceeds by sending an Identify Consumers message for each of
the zero or more individual event IDs identified as being consumed by an Identify Events message 
addressed to the DBC. If there are none of these, this check passes. If there
are one or more, it then checks:

\begin{enumerate}
\item That exactly one Consumer Identified reply is received 
    for each Identify Consumers message sent.
\item That those show the DBC node as their source,
\item That these identify the same event ID as the corresponding Identify message.
\end{enumerate}

Other messages received during this 
test are ignored.

\subsection{Initial Advertisement}

Follow the prompts when asked to reset or otherwise initialize the DBC.

This section's checks the interaction in the preamble to Standard section 6, and
the messages in Standard sections 4.1, 4.3, 4.4, 4.6 and 4.7.

\textbf{Note:}  There's no requirement that the Identify Producer messages
be sent immediately, only that they be sent before the events are produced.
Nodes typically send them immediately, and that's what this checks.

This check starts by restarting the node, which causes a transition to Initialized
state.  That is then followed by the node identifying events that it will 
produce and consume by appropriate messages. This then checks:

\begin{enumerate}
\item That the Producer Identified, Producer Identified Range, Consumer Identified 
    and Consumer Identified Range messages produced at node startup are the same
    as the ones emitted in response to an addressed Identify Events,
\item That those messages show the DBC as their source.
\end{enumerate}

During this process,  any received PCER messages with the DBC as their source are compared against
previously received Producer Identified and Producer Range Identified messages.
If the Event ID carried by the PCER message has not been identified by 
a previous message, the check fails.

\subsection{Events With Payload}

This section checks that PCER Events With Payload can successfully be processed by 
the node. 

Because there is no standard way to force an arbitrary
node to produce or consume PCER Events With Payload, this only checks
that the node can coexist with PCER Events With Payload on the network.

\begin{enumerate}
\item Produce a PCER Event with Payload of 12 bytes, 
    e.g the initial 8 bytes and an additional 4 in a 2nd frame.
\item Produce a PCER Event with Payload of 16 bytes.
\item Produce a PCER Event with Payload of 20 bytes.
\item Produce a PCER Event with Payload of 24 bytes.
\item Produce a PCER Event with Payload of 256 bytes.
\end{enumerate}

After those have been sent, the check waits 30 seconds to see if there are any additional
transmissions, including but not limited to an alias-allocation series or a Node
Initialized message. If any are seen, that indicates the node has reinitialized
and the check fails.

The operator is also prompted to check whether the node indicates that it has
restarted, e.g. via LEDs or a screen.  If it has, the check fails.

\subsection{Learn Event}

This section checks that the DBC is compatible with the
Learn Event messages in Sections 4.9 and 6.5 of the Standard.

Because there is no standard way to force an arbitrary
node to send or process Learn Event messages,
this only checks
that the node can coexist with Learn Event messages on the network.

\begin{enumerate}
\item Send a Learn Event message with 
    an Event ID  formed from the checker's Unique ID (Node ID).
\end{enumerate}

After those have been sent, the check waits 30 seconds to see if there are any additional
transmissions, including but not limited to an alias-allocation series or a Node
Initialized message. If any are seen, that indicates the node has reinitialized
and the check fails.

The operator is also prompted to check whether the node indicates that it has
restarted, e.g. via LEDs or a screen.  If it has, the check fails.

\end{document}  
