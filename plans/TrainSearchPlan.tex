\documentclass[11pt]{article}
\usepackage{geometry}                % See geometry.pdf to learn the layout options. There are lots.
\geometry{letterpaper}                   % ... or a4paper or a5paper or ... 
%\geometry{landscape}                % Activate for for rotated page geometry
\usepackage[parfill]{parskip}    % Activate to begin paragraphs with an empty line rather than an indent
\usepackage{graphicx}
\usepackage{amssymb}
\usepackage{epstopdf}
\usepackage{hyperref}
\DeclareGraphicsRule{.tif}{png}{.png}{`convert #1 `dirname #1`/`basename #1 .tif`.png}

% fornat header abd footer
\usepackage{fancyhdr}
\fancyhf{}
\fancyfoot[L]{\footnotesize{Copyright 2022-2024. All rights reserved. 
        This OpenLCB document is licensed under the 
        \newline
        Creative Commons Attribution-ShareAlike 4.0 International License (CC BY-SA 4.0). 
        \newline
        See https://openlcb.org/licensing for more information.
    }
    \hfill \thepage
}
\renewcommand\headrulewidth{0pt}
\setlength{\headheight}{13.59999pt}

\fancypagestyle{firststyle}
{
   \fancyhf{}
    \fancyfoot[L]{\footnotesize{Copyright 2022-2024. All rights reserved. 
        This OpenLCB document is licensed under the 
        \newline
        Creative Commons Attribution-ShareAlike 4.0 International License (CC BY-SA 4.0). 
        \newline
        See https://openlcb.org/licensing for more information.
        }
        \hfill \thepage
    }
   \renewcommand{\headrulewidth}{0pt} % removes horizontal header line
}

\pagestyle{fancyplain}
% The name is put in the header by the titleandheader command below

% common header content elements
%\author{The OpenLCB Group}
\date{\today}                                         % Activate to display a given date or no date


% specify the text for the title and footer
\newcommand*{\titleandheader}[1]{
    \title{#1}
    \fancyhead[L]{\footnotesize{#1}}
}

% to format XML tags in proper angle brackets <>
\newcommand*{\xml}[1]{\texttt{<#1>}}

% macro for inserting common introduction text
% takes hyperlink to a standard and the standard name as arguments
\newcommand{\introductionCaveats}[2]{

    \section{Introduction}

    This note documents the procedure for checking an OpenLCB implementation against the 
    \href{#1}
    {#2}.

    The checks are traceable to specific sections of the Standard.

    The checking assumes that the Device Being Checked (DBC) is being exercised by other
    nodes on the message network, 
    e.g. is responding to enquiries from other parts of the message network.
}

% macro for inserting the procedure section
\newcommand{\checkProcedure}[1]{
%    Select ``#1" in the program, 
%    then select each section below in turn.  Follow the prompts
%    for when to reset/restart the node and when to check 
%    outputs against the node documentation.
}

% macro deciding whether the pipset footnote should appear or not
\newcommand{\pipsetFootnote}{
%\footnote{Using the -p option or setting the checkpip default value False will skip this check.}
}


\titleandheader{Checking the OpenLCB Train Search Protocol Standard}

\begin{document}
\maketitle
\thispagestyle{firststyle}

\introductionCaveats
    {https://nbviewer.org/github/openlcb/documents/blob/master/standards/TrainSearchS.pdf}
    {Train Search Protocol Standard}

The Train Search Protocol specifies the behavior of both providers and requestors of 
train search information.  Only the provider is checked here.


\section{Train Search Protocol Provider Procedure}

\checkProcedure{Train Search Protocol Provider Checking}

These check procedures require the attached node, when appropriately commanded,
to create a virtual train node with address 12. They may not be appropriate
for e.g. train nodes with a fixed Node ID.

\subsection{Virtual Train Node creation check}

This checks the interaction in section 6.2 of the Standard
followed by checking the interaction in section 6.1 of the Standard.

This check must start with the device being checked having not created a
virtual node with address 12. For some devices, doing a power cycle will 
drop all virtual nodes and prepare the device for this check.

\begin{enumerate}

\item Send a search Identify Producer with search nibbles 0xFF.FF.12 and 
    not-Allocate, Exact, Address Only, Any/Default protocol 0x60.
    
\item Wait one second for a reply.  There should not be one.

\item Send a search Identify Producer with search nibbles 0xFF.FF.12 and 
    Allocate, Exact, Address Only, Any/Default protocol 0xE0.

\item Wait up to one second for a reply.  A Producer Identified Valid for the same event ID
    should be received.

\item Send a search Identify Producer with search nibbles 0xFF.FF.12 and 
    not Allocate, Exact, Address Only, Any/Default protocol 0x60.

\item Wait up to one second for a reply.  A Producer Identified Valid for the same event ID
    should be received.    

\end{enumerate}

\subsection{Match partial values check}

This is checking aspects of the Event Identifier matching algorithm described
in section 6.3 of the Standard.

\begin{enumerate}

\item Ensure node with address 123 exists; may have been created in the previous check.

    \begin{enumerate}
    \item Send a search Identify Producer with search nibbles 0xFF.F1.23 and 
        Allocate, Exact, Address Only, Any/Default protocol 0xE0.

    \item Wait up to one second for a reply.  A Producer Identified Valid for the same event ID
        should be received.
    \end{enumerate}

\item Search not-allocate for partial match on 1; should exist
    \begin{enumerate}
    \item Send a search Identify Producer with search nibbles 0xFF.FF.F1 and 
        not-Allocate, not-Exact, Address Only, Any/Default protocol 0x20.

    \item Wait up to one second for a reply.  A Producer Identified Valid for the same event ID
        should be received.
    \end{enumerate}

\item Search not-allocate for partial match on 12; should exist
    \begin{enumerate}
    \item Send a search Identify Producer with search nibbles 0xFF.FF.12 and 
        not-Allocate, not-Exact, Address Only, Any/Default protocol 0x20.

    \item Wait up to one second for a reply.  A Producer Identified Valid for the same event ID
        should be received.
    \end{enumerate}

\end{enumerate}

(An additional check that 0x23 doesn't match would be good, but is there a good-enough way
to ensure that a locomotive matching that number doesn't exist during the check?)

\subsection{Reserved address values check}

This is checking the reserved values found in Table 2 of section 5.2 of the Standard.

Having 0xA through 0xE in a search nibble should result in no matches.

\begin{enumerate}

\item Ensure node with address 12 exists; may have been created in the previous check.

    \begin{enumerate}
    \item Send a search Identify Producer with search nibbles 0xFF.FF.FF.12 and 
        Allocate, Exact, Address Only, Any/Default protocol 0xE0.

    \item Wait up to one second for a reply.  A Producer Identified Valid for the same event ID
        should be received.
    \end{enumerate}

\item Send a search Identify Producer with search nibbles 0xFF.FF.FC and 
    Allocate, Exact, Address Only, Any/Default protocol 0xE0.

\item Wait one second for a reply.  There should not be one.

\end{enumerate}

\subsection{Reserved address values check}

This is checking the reserved values found in Table 4 of section 5.2 of the Standard.

Having reserved values in the protocol nibble should result in no matches.

\begin{enumerate}

\item Ensure node with address 12 exists; may have been created in the previous check.

    \begin{enumerate}
    \item Send a search Identify Producer with search nibbles 0xFF.FF.12 and 
        Allocate, Exact, Address Only, Any/Default protocol 0xE0.

    \item Wait up to one second for a reply.  A Producer Identified Valid for the same event ID
        should be received.
    \end{enumerate}

\item Send a search Identify Producer with search nibbles 0xFF.FF.12 and 
    protocol byte Allocate plus reserved values 0xF0.

\item Wait one second for a reply.  There should not be one.

\item Send a search Identify Producer with search nibbles 0xFF.FF.12 and 
    protocol byte Allocate plus reserved values 0xF8.

\item Wait one second for a reply.  There should not be one.

\item Send a search Identify Producer with search nibbles 0xFF.FF.12 and 
    protocol byte Allocate plus reserved values 0xE3.

\item Wait one second for a reply.  There should not be one.

\end{enumerate}

\end{document}  
