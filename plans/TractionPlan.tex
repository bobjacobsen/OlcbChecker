\documentclass[11pt]{article}
\usepackage{geometry}                % See geometry.pdf to learn the layout options. There are lots.
\geometry{letterpaper}                   % ... or a4paper or a5paper or ... 
%\geometry{landscape}                % Activate for for rotated page geometry
\usepackage[parfill]{parskip}    % Activate to begin paragraphs with an empty line rather than an indent
\usepackage{graphicx}
\usepackage{amssymb}
\usepackage{epstopdf}
\usepackage{hyperref}

\DeclareGraphicsRule{.tif}{png}{.png}{`convert #1 `dirname #1`/`basename #1 .tif`.png}

% macro for inserting common introduction text
% takes hyperlink to a stadnard and the standard name as arguments
\newcommand{\introductionCaveats}[2]{

    \section{Introduction}

    This note documents the procedure for checking an OpenLCB implementation against the 
    \href{#1}
    {#2}.

    The checks are traceable to specific sections of the Standard.

    The checking assumes that the Device Being Checked (DBC) is being exercised by other
    nodes on the message network, 
    e.g. is responding to enquiries from other parts of the message network.
}

% macro for inserting the procedure section
\newcommand{\checkProcedure}[1]{

    Select ``#1" in the program, 
    then select each section below in turn.  Follow the prompts
    for when to reset/restart the node and when to check 
    outputs against the node documentation.
}



\title{Checking the OpenLCB Traction Protocol Standard}

\begin{document}
\maketitle


\introductionCaveats
    {https://nbviewer.org/github/openlcb/documents/blob/master/drafts/TractionS.pdf}
    {(Draft) Traction Protocol Standard}

\section{Traction Protocol Procedure}

\checkProcedure{Traction Protocol Checking}

A node which does not self-identify in PIP that it supports
the Traction Protocol should be considered to have passed these checks.
\pipsetFootnote

Note that this commands the speed and functions of a traction node.
Although the commanded speed is quite low, the
tests should be run in a way that the associated physical locomotive
does not run away.

\subsection{Defined Event ID checking}

This section checks that the node supports the isTrain event
defined in section 4.1 and 6.4 of the Traction Protocol Standard.

It does this by issuing an Identify Events to the node, and then
checking for a Producer Identified reply carrying the isTrain Event ID.

\subsection{Check set and query speeds}

\subsection{Check query controller}

\subsection{Check set and query functions}

\subsection{Check emergency stop}

\subsection{Check controller configuration command and response}

\subsection{Check memory spaces}

This section checks the memory spaces defined in section 7
of the Traction Protocol Standard.

\footnote{This does not check the information defined by the 
    Function Definition Information Standard.
}
    
\subsection{}

\end{document}  
